\chapter{Especificação da Tarefa T4 - Análise Bibliométrica\label{chap:tarefa:analise:bibliometrica}}

\section{Pontuação da Tarefa}

A tarefa vale 25 pontos, que serão distribuídos conforme os critérios descritos em \ref{sec:tarefa:analise:biblio:pontuacao}.

Para compreender a especificação da tarefa na sua integralidade será preciso que leia o capítulo \ref{chap:pesquisa:bibliografica:analise:bibliometrica}, os slides em  \ref{chap:bibliometrix:slides} e o exemplo em \ref{chap:bibliometria:jhcf}.

\section{Objetivo da Tarefa}

Produzir individualmente uma análise bibliométrica inicial, abordando o entendimento de quais tipos de variáveis tem sido \textbf{usadas em simulações e (ou) experimentos reais, para estudo do fenômeno de interesse simulado pelo exemplo de simulação escolhido pelo estudante}.

A simulação do fenômeno em foco, objeto de seu estudo, será a que foi escolhida por você no \href{https://aprender3.unb.br/mod/forum/view.php?id=769097}{Fórum de discussão sobre a disciplina Computação Experimental}, no tópico \href{https://aprender3.unb.br/mod/forum/discuss.php?d=217110}{Escolha da simulação base para a construção do seu laboratório de experimentos}. 

A análise bibliométrica desenvolvida deve conter textos, figuras e tabelas, conforme o exemplo em \ref{chap:bibliometria:jhcf}.

\section{Produto a Entregar}

Por exemplo, se você escolheu a simulação \textbf{wolf\_sheep}, que visa simular o \textbf{equilíbrio presa-predador} em um ecossistema, a sua análise bibliométrica pode ter como foco questões como as seguintes:
\begin{quote}
    Como tem sido desenvolvidas pesquisas que usam simulação computacional para investigar o \textbf{equilíbrio presa-predador}, especialmente simulação multiagente?
    
    Como o fenômeno na vida real, o \textbf{equilíbrio presa-predador}, tem sido investigado de forma experimental?

    Que variáveis independentes (estímulos, causas) e dependentes (resultados, consequências) são mais comumente observadas nas simulações computacionais, ou mesmo em experimentos reais? 
    

    Quais as estruturas conceituais, intelectuais e sociais das pesquisas nesse tema?
\end{quote}

A sua análise bibliométrica vai buscar respostas a essas questões (ou questões similares) e é o produto da tarefa, que deve estar apresentado num novo capítulo que você vai criar, na parte \ref{part:estudos:exploratorios}, no relatório da turma no Overleaf.

\section{Exemplo de uma análise bibliométrica quase completa}

Veja como referência de trabalho de grande porte um exemplo da análise descrita no 
no capítulo \ref{chap:bibliometria:jhcf}, abordando desde a  Seção \ref{MASSA@jhcf:questoes} até a Seção  \ref{MASSA@jhcf:analise}.

O conteúdo textual do capítulo \ref{chap:bibliometria:jhcf} está registrado no arquivo ``tarefa-jhcf.tex'', no diretório:
\begin{verbatim}
2-Analise-Exploratoria-Dados /tarefas /T4-Pesquisa-Bibliografica /estudantes/
\end{verbatim}

Os dados e imagens usados no capítulo \ref{chap:bibliometria:jhcf} estão depositados no diretório:
\begin{verbatim}
/exploratory-data-analysis /jhcf / PesqBibliogr / SimulacaoMultiagente/
        WoS-20220203
\end{verbatim}

\section{\textit{Dataset} de Análise}

A análise bibliométrica é um tipo de análise exploratória de dados, e como toda análise de dados, é fundamental evidenciar quais foram os dados usados,  isso é, o \textit{dataset}, e como eles foram gerados.

No capítulo de exemplo, o \textit{dataset} usado está em :

\begin{verbatim}
/exploratory-data-analysis /jhcf / PesqBibliogr / SimulacaoMultiagente
        /WoS-20220203/wos6935recs.txt
\end{verbatim}


\section{Etapas da Análise}

A sua análise precisa ser realizada e descrita em cinco etapas, e deve seguir as orientações feitas em \ref{metodo:analise:bibliografica}, com base no detalhamento proposto por \citet{aria_bibliometrix_2017}:
\begin{enumerate}
    \item \textit{Study design} (Planejamento do estudo);

    \item  \textit{Data collection} (Coleta de dados);

    \item \textit{Data analysis} (Análise dos dados);

    \item \textit{Data visualization} (Visualização dos dados representados de forma gráfica, em vários formatos, vários tipos de diagrama);

    \item  \textit{Interpretation} (Interpretação, traçar conclusões, reflexões, sugestões de aprofundamento).
\end{enumerate}

\section{Critérios de Pontuação\label{sec:tarefa:analise:biblio:pontuacao}}

Os 20 pontos da tarefa serão distribuídos conforme o atendimento aos critérios seguintes.

\begin{description}
    \item [2 pontos] A seção \textit{Study design} (Planejamento do estudo) deve apresentar o fenômeno de interesse (simulado e real) com citação a pelo menos duas referências bibliográficas encontradas antes da pesquisa, sendo que uma delas pode ser uma citação à Wikipedia ou a uma página web, e a outra citação a um artigo de revista, conferência ou livro, encontrados antes da busca em bases de dados bibliográficas;
    \item [3 pontos] A seção \textit{Study design} (Planejamento do estudo) deve apresentar suas perguntas de pesquisa tendo em vista os tipos de perguntas de pesquisa feitas em uma pesquisa bibliométrica;
    \item [2 pontos] A seção \textit{Data collection} (Coleta de dados) deve apresentar cuidadosamente a formação da string de busca feita na Web of Science.
    \item [2 pontos] A seção \textit{Data collection} (Coleta de dados) deve apresentar os passos que foram gerados para obter o \textit{dataset} de análise bibliométrica que será analisado
    \item [2 pontos] A seção \textit{Data collection} (Coleta de dados) deve apresentar um \textit{dataset} que  contem, minimamente, 500 registros bibliográficos, sendo um valor ideal do número de registros do \textit{dataset} estar entre 1.000 e 2.500.
    \item [4 pontos] Na seção \textit{Data visualization} (Visualização dos dados) devem ser apresentados pelo menos três gráficos e três tabelas, para explorar cada uma das três estruturas do conhecimento analisadas pelo Bibliometrix: 
    \begin{itemize}
        \item Estrutura Conceitual do Conhecimento;
        \item Estrutura Social  do Conhecimento e;
        \item Estrutura Intelectual do Conhecimento;
    \end{itemize}
    \item [5 pontos] Na seção   \textit{Interpretation} (Interpretação) todas as perguntas de pesquisa formuladas no início do estudo devem ser avaliadas quanto ao atingimento de respostas, e essas respostas precisam citar e apresentar pelo menos cinco referências bibliográficas relevantes  que estão presentes no \dataset\ explorado. Essas referências devem ser citadas por meio de referências presentes na base de dados do grupo RESIC.
    \item [-10 pontos] Organização do \LaTeX

Na edição do \LaTeX~ deve-se atentar aos seguintes aspectos obrigatórios, e possibilidades de penalidades por descumprimento:
\begin{enumerate}
    \item [-5 pontos] O tema de sua análise bibliométrica deve envolver o uso de simulação do fenômeno simulado no código por você escolhido, e opcionalmente experimentação no mundo real para compreensão do fenômeno. 
    \item [-10 ponto] Todos os dados, inclusive o \textit{dataset} bibliográfico e as imagens, usadas na produção da análise, deve estar inseridos no diretório de análise exploratória de dados, individual do estudante;
    \item [-5 ponto] Todas as figuras e gráficos inseridos na análise devem ser individualmente rotulados com \textit{label}, sem conflitar com os \textit{labels} já existentes, devem ter um título (\textit{caption}) descritivo do que apresenta a figura e o nome do \textit{dataset} usado, e também a figura/gráfico deve ser explicitamente descritas e citadas, usando referência (ref);
    \item [-5 ponto] As figuras deve ser automaticamente dimensionadas, e eventualmente rotacionadas,  para se adaptar à largura e (ou) altura do texto na página, visando ter uma boa visibilidade. 
\end{enumerate}

\end{description}

\section{Entrega da Tarefa}

\subsection{Criando os arquivos}

O nome do arquivo de texto á ser criado com a sua análise bibliométrica deve ter a forma:

\begin{verbatim}
tarefa-<githubusername>.tex
\end{verbatim}

Todas as figuras, tabelas, dados etc, incluídas no texto, devem estar montadas no diretório:
\begin{verbatim}
/exploratory-data-analysis /<githubusername> / PesqBibliogr / <tema-pesquisa>
\end{verbatim}

onde
\texttt{<tema-pesquisa>} é um nome curto, em formato \textit{CamelCase}, que você dará ao diretório onde os dados, figuras, etc, do seu estudo, estarão montados. Não use caracteres acentuados em nomes de arquivos no Overleaf.

Veja exemplos de dados e figuras usadas na preparação do estudo no capítulo \ref{chap:bibliometria:jhcf}, no diretório
\begin{verbatim}
/exploratory-data-analysis /jhcf /PesqBiblogr /SimulacaoMultiagente
\end{verbatim}

O arquivo de texto com o seu capítulo deve ser montado no diretório:
\begin{verbatim}
3-Analise-Exploratoria-Dados / tarefas / T6-Pesquisa-Bibliografica 
    /estudantes /
\end{verbatim}

Depois, você vai fazer o \textit{input} do arquivo de texto em:
\begin{verbatim}
3-Analise-Exploratoria-Dados /tarefas /T6-Pesquisa-Bibliografica 
    /estudantes /main.tex
\end{verbatim}

\subsection{Compilando a Tarefa}

Após fazer o \textit{input} do conteúdo produzido por você, faça os ajustes necessários para garantir que a compilação do \LaTeX~ ocorre sem introdução de erros ou novos \textit{warnings}.

\subsection{Entregando a Tarefa no Github}

Quando a compilação da sua entrega estiver satisfatória, use a interface do Overleaf para enviar suas contribuições para o repositório \textbf{origin}, com a opção:

\begin{verbatim}
Menu -> Sync -> github -> Push Overleaf Changes do Github
\end{verbatim}

Informe no \textit{commit} a seguinte mensagem: 
\begin{verbatim}
Tarefa 6 - <Seu Nome Completo> - <githubusername>
\end{verbatim}

Após o \textit{commit} bem sucedido, use os comandos \verb|git pull; git log| ou equivalente, para ver o \textit{tag} de seu \textit{commit} na lista de \textit{commits} no repositório \textbf{origin}. Informe esse \textit{tag} no texto online desta tarefa. 
Veja um exemplo a seguir:

\begin{verbatim}
commit 3af590b7fc3b06ec4375fb7eaba3b07abb21cda1 
     (HEAD -> main, origin/main, origin/HEAD)

Author: JORGE H C FERNANDES <jhcf@unb.br>

Date:   Tue Jan 25 16:18:27 2022 -0300

    Tarefa 6 - Jorge Henrique Cabral Fernandes - jhcf
\end{verbatim}

\subsection{Registrando a entrega da Tarefa no Aprender}

O \textit{tag} do \textit{commit} deve ser informado no texto online da tarefa no Moodle, como exemplo:
\begin{verbatim}
3af590b7fc3b06ec4375fb7eaba3b07abb21cda1
\end{verbatim}

