\chapter{Análise Bibliográfica sobre simulação da disseminação de um vírus em redes relacionais, por Gustavo Rodrigues Gualberto\label{chap:bibliometria:gustavorgygo}}

\section{Planejamento do estudo\label{gustavorgygo:questoes}}

No caso do meu trabalho, as perguntas que o nortearam foram:

\begin{itemize}

    \item Como são usadas as simulações de infecção viral? 
    
    \item Quais são as variáveis independentes e dependentes que tem sido usadas para o estudo do de infecções virais? 
    
    \item Qual a estrutura social da comunidade que pesquisa sobre o tema?
    
\end{itemize}

\subsection{O que já existe de pesquisa bibliométrica sobre esse tema?}

 Este tema está deveras suscitado perante à comunidade científica em razão da pandemia de COVID-19 que fora responsável pela catástrofe a qual presenciara-se durante o ano de 2020, principalmente \citep{maheshwari_network_2020}. Este fato histórico gerou uma onda de descrença acerca das vacinas e, consequentemente, também fora responsável pelo ressurgimento de casos de doenças praticamente erradicadas, tais como a pólio, por exemplo. \citep{mckeever_poliomielite_2022}. Nesse caso, estarei utilizando como base de estudo o espalhamento de vírus disseminados entre computadores na internet.

 \subsection{Uso do Bibliometrix e Biblioshiny}

 Usufruí a ferramenta biblioshiny e o \textit{workflow} fornecidos pelos desenvolvedores do Bibliometrix \cite{aria_bibliometrix_2017}.

\subsection{Limitações} 

A tarefa relatada foi feito em um dia.

\section{Coleta de dados\label{gustavorgygo:coleta}}

A coleta de dados feita usando o Web of Science (WoS) no dia 07 de Dezembro de 2022, acessado por meio do Portal de Periódicos da CAPES.
 

