\chapter{Análise Bibliográfica sobre Simulação Predador-Presa, por Marco Antônio Souza de Athayde\label{chap:bibliometria:masathayde}}

\section{Planejamento do estudo\label{PPS@masathayde:questoes}}
O trabalho abordará a situação da pesquisa científica sobre simulações da dinâmica natural entre predadores e presas, com utilização de técnicas de análise bibliométrica, para a resolução das perguntas norteadoras escolhidas. O resultado desta tarefa servirá como fundação para um trabalho futuro, no qual será implementada uma simulação multiagente do fenômeno natural de interesse.

Predação é uma forma de interação entre organismos, na qual um, o predador, mata e consome o outro, predador, como alimento. \cite{noauthor_predation_nodate} Com o propósito de estudar esse sistema, cientistas desenvolveram modelos para criar simulações, por meio das quais o fenômeno poderia ser examinado de forma controlada. Em \cite{holling_characteristics_1959}, enfatiza-se a necessidade de primeiro estabelecer um modelo que simule de forma simples a resposta funcional ao consumo de presas, a partir do qual podem ser construídos modelos mais complexos.

No caso do meu trabalho, as perguntas que o nortearam foram:
\begin{itemize}
    \item Qual é o atual estado da pesquisa científica sobre simulação de modelos predador-presa?
    \item Quais são as publicações mais influentes e relevantes sobre o assunto?
    \item Quais são os principais autores que tratam sobre o tema?
    \item Quais são as palavras-chaves relevantes ao tema de simulação predador-presa?
    \item Quais são as publicações mais importantes para o tema?
\end{itemize}

\section{Coleta de dados\label{PPS@masathayde:coleta}}

Coletaram-se referências do banco de dados Web of Science, através do Portal de Periódicos da CAPES.

%Foram feitas buscas nas coleções \textbf{Science  Citation  Index  Expanded (SCI -EXPANDED)} e \textbf{Social  Sciences  Citation  Index (SSCI)}, que contém registros relativos a vários campos do conhecimento, no qual o SCI-EXPANDED foca mais na área das ciências exatas e naturais, enquanto que o SSCI indexa artigos da área das ciências sociais. Observe que os artigos nessas duas coleções são indexados desde 1945. 

\subsection{Query de Busca}

A \query\ de busca utilizada é mostrada a seguir.

\lstinputlisting[numbers=left,basicstyle=\normalsize\ttfamily,caption={\query\  de busca sobre simulação de interação predador presa.},label=query:masathayde:predator:prey]
{exploratory-data-analysis/masathayde/PesqBibliogr/query.txt}

