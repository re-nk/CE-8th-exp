\documentclass[12pt]{book}
%https://www.overleaf.com/project/629e18789b6baeb860fd2545
\input{packages}
\input{packages-estudantes}

% define cores personalizadas para o texto de cada autor
\definechangesauthor[name={Jorge Henrique Cabral Fernandes}, color=orange]{jhcf} % git-user: jhcf
\definechangesauthor[name={Gabriel da Silva Corvino Nogueira}, color=orange]{nosgueira} % git-user: nosgueira
\definechangesauthor[name={Gabriel Ligoski}, color=blue]{gabrielligoski} % git-user: gabrielligoski
\definechangesauthor[name={Gustavo Rodrigues Gualberto}, color=green]{GustavoRGYGO} % git-user: GustavoRGYGO
\definechangesauthor[name={Edgar Sampaio de Barros}, color=green]{edgarsamp} % git-user: edgarsamp
\definechangesauthor[name={Erick Rodrigues Fraga}, color=green]{mutenesss} % git-user: mutenesss
\definechangesauthor[name={João Pedro de Sousa Soares Martins}, color=green]{jpssm} % git-user: jpssm
\definechangesauthor[name={Thiago Masera Tokarski}, color=green]{Toskosz} % git-user: Toskosz
\definechangesauthor[name={Marcelo Aiache Postiglione}, color=green]{marcelo3101} % git-user: marcelo3101
\definechangesauthor[name={Arthur Barreiros de Oliveira Mota}, color=purple]{BarreirosM} % git-user: BarreirosM
\definechangesauthor[name={Weliton César Pereira Barreto}, color=red]{welitonbarreto} % git-user: welitonbarreto
\definechangesauthor[name={Gabriel Pinheiro da Conceicao}, color=blue]{pinheirogh} % git-user: pinheirogh
\definechangesauthor[name={Breno Ribeiro Corrêa}, color=red]{brenoZ} % git-user: brenoZ
\definechangesauthor[name={João Victor de Souza Calassio}, color=blue]{jvcalassio} % git-user: jvcalassio
\definechangesauthor[name={Matheus Barbosa Santos}, color=cyan]{MatBSa} % git-user: MatBSa
\definechangesauthor[name={Plínio Candide Rios Mayer}, color=green]{PlinioMayer} % git-user: PlinioMayer
\definechangesauthor[name={Marco Antonio Nemetala Garcia}, color=green]{Hadeshade} % git-user: Hadeshade
\definechangesauthor[name={Regina Emy Da Nobrega Kamada}, color=orange]{bananaMoshpit} % git-user: bananaMoshpit
\definechangesauthor[name={Felipe Gomes Paradas}, color=purple]{fparadas} % git-user: fparadas
\definechangesauthor[name={Paulo Alvim Alvarenga}, color=yellow]{alvimpaulo} % git-user: alvimpaulo
\definechangesauthor[name={Marco Antônio Souza de Athayde}, color=pink]{masathayde} % git-user: masathayde
\definechangesauthor[name={João Antônio Desiderio de Moraes}, color=green]{joaoadm94} % git-user: joaoadm94
\definechangesauthor[name={Lucas de Oliveira Silva}, color=pink]{LuscasOSilva} % git-user: LuscasOSilva
\definechangesauthor[name={Isaque Augusto da Silva Santos}, color=pink]{seraphritt} % git-user: seraphritt
\definechangesauthor[name={Claudio Roberto Oliveira Peres de Barros}, color=pink]{ClaudioBarros} % git-user: ClaudioBarros
\definechangesauthor[name={André Cássio Barros de Souza}, color=pink]{andreloff} % git-user: andreloff
\definechangesauthor[name={Fernando Ferreira Cordeiro}, color=green]{FernandoCordeiro} % git-user: FernandoCordeiro
\definechangesauthor[name={Paulo Mauricio Costa Lopes}, color=orange]{RequiemDosVivos} % git-user: RequiemDosVivos
\definechangesauthor[name={Bruno Abreu Kamienski}, color=magenta]{brunosabreu} % git-user: brunosabreu
\definechangesauthor[name={André Fioravante Nicolodi Durante Júnior}, color=green]{AndreDuranteJr} % git-user: Andredurantejr

\definechangesauthor[name={Gabriel Teixeira Ribeiro}, color=blue]{moltentheory} % git-user: moltentheory

%MILLER, John H.; PAGE, Scott E. Complex Adaptive Systems: An introduction to computational models of social life. USA: Princeton University Press, 2007.


 

\makenoidxglossaries
\loadglsentries{1-Introducao/tarefas/T2-Glossario/estudantes/main}

\setcounter{tocdepth}{5}
\setcounter{secnumdepth}{5}
%\captionsetup[table]{name=Quadro}
\renewcommand{\lstlistingname}{Listagem de Código}

\newcommand{\dataset}{\textit{dataset}}
\newcommand{\query}{\textit{query}}
\newcommand{\githubusername}{\textless githubusername\textgreater}

\def\printpart{1} % Orientações Gerais, Introdução e Fundamentos de Pesquisa Bibliográfica

% Substitua o valor de printpart por 1, 2, 3, 4 ou 5, conforme a parte do documento que você quer gerar 
\def\printpart{1}

\setlength{\headheight}{73.04742pt}
\addtolength{\oddsidemargin}{-1.5cm}
\addtolength{\evensidemargin}{-1.5cm}
\addtolength{\textwidth}{3cm}
\addtolength{\topmargin}{-3cm}
\addtolength{\textheight}{2cm}
	
\begin{document}
\pagenumbering{gobble}% Remove page numbers (and reset to 1)
\clearpage
\thispagestyle{empty}

\begin{titlepage}
\begin{center}
 {\huge\bfseries \includegraphics[width=4cm]{unb-logo.jpg}\\
	CIC0203 - Computação Experimental - TA - 2022.1 - Notas e Registros\\ Compilação individual de jhcf\\
	}

 % ----------------------------------------------------------------
 \vspace{1.5cm}
{\large	% author names and affiliations
	Jorge Henrique Cabral Fernandes (jhcf)\\}
	\vspace{1.5cm}
	{\large Brasília, \DTMnow}
\end{center}
\end{titlepage}
	%\date{10 de março de 2021}
	% make the title area
%	\maketitle
    \listoftodos
	\printnoidxglossary
	\tableofcontents
	\listoffigures
	\listoftables
	\clearpage
\pagenumbering{arabic}

\pagestyle{fancy}

\part{Compilação individual\label{part:intro}}

\chapter{Tarefa 8 - Teste Estatístico de Hipóteses - Parte 1. 15 pontos\label{tarefa:teste:hipotese:R}}

Antes de realizar essa tarefa, veja atentamente as instruções do capítulo anterior: "Conceitos Centrais em Estatística Inferencial".

Nessa tarefa você vai criar um novo capítulo, a partir do desenvolvido na tarefa anterior, com um conjunto de mais seis seções, após a atual seção "Conclusão" do capítulo do seu laboratório, conforme as seguintes orientações:
%como no modelo da seção \ref{tarefa:teste:hipotese:R:jhcf}, contendo um %texto com as seguintes características:
\begin{enumerate}
    \item Renomeie a atual seção "Conclusões" do seu capítulo, para "Conclusões Preliminares"; 
    \item Acrescente as seguintes seções ao seu capítulo:
    \begin{description}
        \item [Apresentação das Amostras] Nessa seção, refine o que antes era apenas a apresentação das "Variáveis Dependentes e Independentes", e realize a apresentação das amostras. 
        Descreva cuidadosamente as amostras coletadas, apresentando cada uma das colunas da planilha de amostras de experimentos por você coletadas, e faça uma descrição detalhada de cada coluna, que contenha:
        \begin{itemize}
            \item O nome da variável;
            \item O que significa a variável;
            \item Se a variável é para uso interno do simulador, variável de controle no experimento, variável independente ou variável dependente;
            \item As faixas de valores ou categorias dos registros obtidos, para fins de interpretação dos números (ex: Se umidade = 0, o que isso significa? Se umidade = 1, o que significa?;
        \end{itemize}
        
        Além disso, apresente os gráficos de distribuição de frequências para pelo menos quatro conjuntos de registros que contenham os mesmos valores para as variáveis independentes. Cada um desses conjuntos de registros será doravante chamados de Amostra.
        
        \item [Primeira Declaração Formal de Hipóteses] Supondo que foi feita a escolha inicial do teste t de Student, para comparar médias entre duas amostras (conjunto de registros) de variável dependente, que foram obtidas a partir de estímulos com diferentes valores para pelo menos uma variável independente, faça uma declaração de hipóteses nula e alternativa para o seu teste. Aplique corretamente os conceitos de hipóteses nula e alternativa, conforme o vídeo sobre Inferência Estatística.
        \item [Aplicação do Teste t] Use as instruções em \url{https://stat.ethz.ch/R-manual/R-devel/library/stats/html/t.test.html}, e com base no vídeo sobre teste t (\url{https://www.youtube.com/watch?v=AgDC9yoopUA}), apresente duas diferentes execuções do teste, comparando diferentes amostras. Apresente, usando lstlistings, todo o código R usado, os resultados apresentados no Console do R, e sua interpretação dos resultados. Informe sobre a comprovação ou rejeição da hipótese nula;
        \item [Segunda Declaração Formal de Hipóteses] Supondo que agora vais fazer uma análise de regressão linear, com as mesmas amostras, veja o vídeo e o texto sobre regressão linear em R:
        \begin{itemize}
            \item \url{https://www.youtube.com/watch?v=u1cc1r_Y7M0}; e
            \item \url{https://www.r-tutor.com/elementary-statistics/simple-linear-regression/significance-test-linear-regression}.
        \end{itemize}
        Com base nos vídeos, faça uma declaração de hipóteses nula e alternativa para o seu teste, que estabeleça possível relação linear entre uma variável independente e uma dependente. Aplique corretamente os conceitos de hipóteses nula e alternativa.
        \item [Aplicação do teste de regressão linear] Apresente duas diferentes execuções do teste, comparando diferentes amostras por meio de regressão linear. Apresente todo o código R dos comandos e dos resultados apresentados no Console do R, e sua interpretação dos resultados, sobre a comprovação ou não da hipótese nula;
        \item [Conclusões] Apresente suas conclusões sobre os testes realizados.
    \end{description}
 \end{enumerate}
    


\bibliographystyle{plainnat}
\bibliography{RESIC}

\end{document}