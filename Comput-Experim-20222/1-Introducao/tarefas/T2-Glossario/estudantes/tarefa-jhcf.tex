\newglossaryentry{SI}{
	name={Sistema de Informação},
	description={Uma combinação coesiva de processos que tratam da coleção, transformação, armazenamento e recuperação de dados, que contém novidades informativas para usuários, quaisquer que sejam os meios tecnológicos utilizados. Fonte: \citep[p. 38]{prakken_information_2000} (Tradução livre)}}

\newglossaryentry{tecnologia}{
	name={Tecnologia (Technology)},
	description={Todas as ferramentas, máquinas, utensílios, armas, instrumentos, moradias, vestimentas, dispositivos de comunicação e transporte, combinados com as habilidades pelas quais esses elementos são produzidos e utilizados. Fonte: \citet[p. 860]{bain_technology_1937}} (Tradução livre)}


\newglossaryentry{EA}{
	name={Experimentação Algorítmica (\textit{Experimental Algorithmics})},
	description={A experimentação algorítmica, ou algorítmica experimental, trata algoritmos como assunto de laboratório, enfatizando o controle de parâmetros [algorítmicos], isolamento de componentes-chave [do algoritmo], construção de modelos e análise estatística. Fonte: \citet[p. 3]{mcgeoch_experimental_2007}} (Tradução livre)}

\newglossaryentry{Glossario}{
    name = {Glossário},
    description  = {Um glossário é uma lista alfabética de termos de um determinado domínio de conhecimento com a definição destes termos. Tradicionalmente um glossário aparece no final de um livro e inclui termos citados que o livro introduz ao leitor ou são incomuns. Fonte: \cite{wikipedia_glossario_2019}.}}