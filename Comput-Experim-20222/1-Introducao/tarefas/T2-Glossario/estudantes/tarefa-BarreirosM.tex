\newglossaryentry{BaseEmpirica}{
    name = {Base Empírica},
    description  = {\citet{ancona_lopez_consideracoes_2011} disse: ``A base empírica, assim como o conhecimento empírico, vai variar de pessoa para pessoa.  Se o mesmo método for aplicado a um mesmo objeto, por pessoas diferentes, o resultado será diferente. Este resultado é a base empírica da pesquisa. Portanto a base empírica é mutável pela mudança do pesquisador, pela mudança de métodos e, claro, pela mudança de objeto.Uma pesquisa, portanto, começa sem a base empírica em si, mas apenas com um indicativo de seus instrumentos (método e objeto).Para alguns, sobre começar a pesquisa sem base empírica, segundo primeira interpretação do assunto, entende-se que é possível você começar sem a base empírica, uma vez que você vai criar, por exemplo, questionários ou roteiros de entrevista para o levantamento dos dados que serão analisados.'' Fonte: \citep{ancona_lopez_consideracoes_2011} \todo{jhcf: Não há uma \textbf{definição} para base empírica no seu item de glossário, como seria esperado}.}}