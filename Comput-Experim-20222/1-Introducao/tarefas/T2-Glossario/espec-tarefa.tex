\chapter{Tarefa 2: Criar um item no glossário deste documento \label{tarefa:glossario}}

\section{Instruções}

\subsection{Fundamentos}

Um glossário é uma lista alfabeticamente ordenada de termos de difícil compreensão, ou significado técnico especializado, pertinentes. Atua como instrumento de auxílio à compreensão de 
um determinado conjunto de conhecimentos ou documentos.
 
\subsection{A Tarefa}

Registre, no glossário do relatório da disciplina Computação Experimental, uma definição de um ou mais termos relacionados com o conteúdo da discussão sobre o que é a ciência (\ref{ciência:o:que:eh}), nele incluído o texto ``Considerações Preliminares sobre a Ciência'', de autoria do professor, em:
\begin{verbatim}
1-Introducao /aulas /Ciencia-e-sua-Avaliacao.pdf
\end{verbatim}

O texto do item de glossário que você vai criar deve ser escrito em língua portuguesa, deve estar claro e compreensível, e conter uma definição para o termo e, quando pertinente, um exemplo de caso concreto.

A definição deve referenciar pelo menos um item bibliográfico presente no Grupo Zotero RESIC, que contenha título, autor e data.


O diretório onde o arquivo com a tarefa deve ser criado é: 
\begin{verbatim}
1-Introducao /tarefas /T2-Glossario /estudantes    
\end{verbatim}

O nome do arquivo á ser criado deve ter a forma:
\begin{verbatim}
tarefa-<githubusername>.tex
\end{verbatim}

O texto do arquivo deve seguir o modelo da tarefa exemplo do professor, em:
\begin{verbatim}
1-Introducao /tarefas /T2-Glossario /estudantes /tarefa-jhcf.tex
\end{verbatim}

Depois de criar o texto do arquivo com o seu item de glossário,faça o input do mesmo no arquivo:
\begin{verbatim}
1-Introducao /tarefas /T2-Glossario /estudantes /main.tex
\end{verbatim}

\subsection{Compilando a Tarefa}

Após fazer o \textit{input} do conteúdo produzido por você, faça os ajustes necessários para garantir que a compilação do \LaTeX~ ocorre sem introdução de erros ou novos \textit{warnings}.

\subsection{Entregando a Tarefa no Github}

Quando a compilação da sua entrega estiver satisfatória, use a interface do Overleaf para enviar suas contribuições para o repositório \textbf{origin}, com a opção:

\begin{verbatim}
Menu -> Sync -> github -> Push Overleaf Changes do Github
\end{verbatim}

Informe no \textit{commit} a seguinte mensagem: 
\begin{verbatim}
Tarefa 2 - <Seu Nome Completo> - <githubusername>
\end{verbatim}

Após o \textit{commit} bem sucedido, use os comandos \verb|git pull; git log| ou equivalente, para ver o \textit{tag} de seu \textit{commit} na lista de \textit{commits} no repositório \textbf{origin}. Informe esse \textit{tag} no texto online desta tarefa. 
Veja um exemplo a seguir:

\begin{verbatim}
commit 3af590b7fc3b06ec4375fb7eaba3b07abb21cda1 
     (HEAD -> main, origin/main, origin/HEAD)

Author: JORGE H C FERNANDES <jhcf@unb.br>

Date:   Tue Jan 25 16:18:27 2022 -0300

    Tarefa 2 - Jorge Henrique Cabral Fernandes - jhcf
\end{verbatim}

\subsection{Registrando a entrega da Tarefa no Aprender}

O \textit{tag} do \textit{commit} deve ser informado no texto online da tarefa no Moodle, como exemplo:
\begin{verbatim}
3af590b7fc3b06ec4375fb7eaba3b07abb21cda1
\end{verbatim}

\subsection{Pontuação pela execução da  Tarefa}

A sua resposta a essa atividade vale até 3\% da pontuação total da disciplina.

\todo[inline]{Os usuários Williamcs1400 e Xavier-Edups usaram o mesmo código RB para definição do seu item de glossário. O último que inseriu o código em conflito deve fazer os ajustes.}