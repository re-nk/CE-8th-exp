%\addbibresource{tarefa-masathayde.bib}

\section{Minhas impressões iniciais sobre a ciência, por Marco Antônio Souza de Athayde}

A ciência não necessariamente segue um caminho contínuo, ininterrupto, previsível. Mesmo com o rigor e a eficiência da metodologia científica, ela, até o momento, é incapaz de perfeitamente modelar todos os fenômenos, naturais ou não.

Em vários momentos da história, a comunidade científica encontrou-se em crise. Isso ocorre quando o conhecimento corrente alcançou seu limite, e cientistas não conseguem criar modelos satisfatórios para solucionar os problemas mais relevantes. Nesses momentos, somente por meio de uma \gls{revolucaocientifica}, a ciência avança e supera os obstáculos intelectuais. Novos paradigmas são estabelecidos, e a ciência reorganiza-se, mais preparada e resistente.

Sem dúvida, dentre todas as áreas de atuação humana, talvez a ciência seja a que reúne os indíviduos mais preparados para uma quebra de paradigma. Realizar ciência é, pois, uma revelação de limitação e vieses pessoais. O cientista deve desapegar-se de tudo que atrapalhe a busca pelo conhecimento e estar sempre pronto para deixar de lado o conhecimento anterior, caso seja demonstrado que este não é mais adequado.

Ainda assim, a ciência está sujeita a influências deturpadoras, como interesses pessoais e financeiros, como descrito em \citep{chavalarias_whats_2017}. Por isso, a comunidade científica deve continuar a esforçar-se para manter sua integridade, buscando sempre o ideal da prática científica.