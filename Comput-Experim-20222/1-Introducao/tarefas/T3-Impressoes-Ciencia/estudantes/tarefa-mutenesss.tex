\section{Minhas impressões iniciais sobre a ciência, por Erick Rodrigues Fraga}

A ciência é um modo sistemático de criar, adquirir e registrar conhecimentos, sejam eles específicos ou mais gerais. Esse conhecimento gerado pela ciência é utilizado para permitir que haja progresso na sociedade, com a resolução de problemas que afetam a todos nós, com o surgimento de novas tecnologias, e consequentemente, novos conhecimentos.\\

Uma das formas de registrar a ciência é utilizando a escrita, de forma que esse registro vai ser utilizado para comprovar aquele conhecimento ou apresentar onde se encontram as falhas presentes nos testes. Para isso, pode ser utilizado um padrão de escrita conhecido como \gls{EscritaAcademica}.\\

Porem, é preciso ter cuidado na forma como é efetuada esse registro, de forma que é possível complicar demais uma explicação ao tentar abordar todos os pontos de falha em um argumento, como apresentado no vídeo \citep{answer_in_progress_why_2022}.
\todo{jhcf: Referência sem nome de autoria.}

Independentemente da complicação de um registro, o conhecimento gerado por esse registro e os subsequentes que se baseiam levam a ciência para novos locais, criando assim novos conhecimentos e fomentando o progresso da ciência.
\todo{jhcf: e da humanidade?.}