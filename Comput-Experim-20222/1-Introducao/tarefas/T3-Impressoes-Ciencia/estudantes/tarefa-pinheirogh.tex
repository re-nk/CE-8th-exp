\section{Minhas impressões iniciais sobre a ciência, por Gabriel Pinheiro da Conceição}

Os seres humanos sempre estiveram rodeados de dúvidas em relação à sua existência e a existência outros múltiplos fenômenos que norteiam a vida terrestre. Segundo \citet{gnipper_o_nodate}, ``a ciência é aquele tipo de conhecimento que busca compreender verdades ou leis naturais para explicar o funcionamento das coisas e do universo em geral''. E para além de todos os questionamentos pertinentes surgiram desde o entendimento do próprio humano com ser pensante diversas formas teorizadas de tentar-se resolver tais questões.

Muitos dos métodos desenvolvidos não conseguiram se provar efetivos em suas proposições a cerca, principalmente, da produção de conhecimento. No entanto, o método científico de investigação através da experimentação, teorização e replicação dos tais por diferentes cientistas e linhas de estudo até hoje se prova ser o caminho de maior sucesso na produção de conhecimento. Graças à \gls{Cientometria}, ciência que estuda o aumento do progresso na resolução dos problemas propostos em investigações utilizando a ciência, hoje, sabemos que o método científico é amplamente utilizado e altamente confiável.