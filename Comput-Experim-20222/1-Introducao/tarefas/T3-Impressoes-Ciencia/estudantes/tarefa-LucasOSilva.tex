\section{Minhas impressões iniciais sobre a ciência, por Lucas de Oliveira Silva}

Após alguns anos convivendo com vários estudos sobre os mais variados tipos durante o ensino médio em uma universidade federal, em diversas matérias, e agora na UnB tive contato com esse assunto poucas vezes e acredito que apenas agora posso começar a ter minha concepção sobre o que é a ciência e meus primeiros pensamentos em relação a ela. A ciência se baseia e consiste em estudo e a forma como estudamos qualquer tipo ou área do conhecimento, afim de expandir ou aprimorar o conhecimento humano, como já vi no ensino médio, em filosofia, tudo se origina da própria filosofia, desde a matemática até sociologia, tudo se tem origem na antiga e única filosofia, que era a unificação de todo o conhecimento, que devido ao grande conhecimento, foi ramificada e denominada em diferentes áreas, é interessante notar que a ciência já tem sua complexidade desde o seu próprio conceito e aplicação, com métodos que precisam ser bem definidos e dependendo também da \gls{Universalidade}, reproducibilidade e projetização de sua aplicação, dentre outros conceitos que definem a ciência e o método cientifico.

Devido a complexidade de seu conceito e aplicação, devemos tomar muito cuidado com a maneira que tudo é feito, para acharmos a melhor teoria possível, com provas validas, e mesmo sendo assim durante toda a história tivemos varias repaginações de teorias que foram invalidadas por outras descobertas, um conceito que ajudada na construção de tudo é a cientometria \citep{vinkler_evaluation_2010}.
\todo{jhcf: Cientometria é um conceito? Ajuda de que forma?} 