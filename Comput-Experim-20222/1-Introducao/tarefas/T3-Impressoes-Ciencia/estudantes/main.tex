% Aqui segue um exemplo de input para o registro das impressões sobre a ciência, criadas pelo professor

\section{Minhas impressões iniciais sobre a ciência, por Jorge Henrique Cabral Fernandes}


Tendo já uma vivência de algumas décadas sobre o que é a ciência, abordarei de modo mais específico minhas impressões sobre a Ciência da Computação \citep{baldwin_three-fold_1994}, que é uma ciência específica dentre as várias existentes. A Ciência da Computação apresenta-se muito relacionada com a filosofia da ciência \citep{floridi_blackwell_2004}, ou como uma meta-ciência (A filosofia é um campo do pensamento humano que é base para a ciência). Um exemplo que comprova essa versatilidade - ou até mesmo universalidade - da Ciência da Computação é o emprego da computação para a criação de simulações como método de pesquisa empírica, isso é o uso de simulações computacionais como base para a pesquisa fundamentada em coleta e análise de dados \citep{tedre_experiments_2014}. A versatilidade do computador, sendo uma Máquina de Turing, possibilita o seu uso para gerar infinitas plataformas empíricas para coleta de dados, pois os computadores são prodigiosos na coleta, cálculo e geração de dados. Assim sendo, pode-se usar o computador para o desenho e execução de quase todo tipo de experimento científico, em quaisquer dos outros campos do conhecimento. Situações mais específicas podem ser vistas no campo da \gls{EA}.

Posto isso, refletir sobre o uso de simulações computacionais possibilita também refletir muito sobre o que é a ciência, e portanto, possibilita uma abordagem filosófica.


\newglossaryentry{Cientometria}{
	name={Cientometria},
    description={Cientometria é definida como o estudo da mensuração e quantificação do progresso científico, estando a pesquisa baseada em indicadores bibliométricos. A cientometria tem um grande potencial de aplicação, havendo interesse de Governos e instituições de pesquisas em utilizar este conhecimento com o objetivo de implementar diferentes formas de apoio ao desenvolvimento científico e tecnológico. Fonte: \cite{dasilva_cientometria_2001}}}

\section{Minhas impressões iniciais sobre a ciência, por Marcelo Aiache Postiglione}

A ciência pode representar tanto uma área de conhecimento quanto um processo. Em meu entendimento, a ciência como área do conhecimento é o conjunto de toda a sabedoria adquirida sobre aquele escopo específico (Ciência da computação, Ciências biológicas, Ciências Sociais, etc). Como processo, a ciência é uma atividade humana que segue passos sistemáticos com o objetivo de adquirir ou aprofundar o conhecimento sobre algo. Esses passos são usados, por exemplo, para levantar dados, criar uma \gls{Hipotese} ou formular uma nova teoria. É uma atividade guiada pelo método científico e outros princípios, que visa estudar, entender e registrar os mais diversos fenômenos do universo e assim, tentar  interligar fatos isolados com o objetivo de promover uma melhor compreensão sobre o mundo e a natureza. 

Essa busca sistemática pelo conhecimento, nos permitiu grandes avanços e sua contribuição para a sociedade é inegável. Por exemplo, durante a pandemia da COVID-19 tivemos vários estudos importantes para entendermos melhor como a doença se manifestava e até mesmo suas implicações no coração \cite{strabelli_covid-19_2020}, além de todos os trabalhos que culminaram na criação das vacinas.

\newglossaryentry{scientific_journals}{
	name={Scientific journals (Revistas científicas)},
	description={As revistas científicas representam o meio mais vital para divulgar os resultados da pesquisa e geralmente são especializadas em diferentes disciplinas ou subdisciplinas acadêmicas. Muitas vezes, a pesquisa desafia suposições comuns e/ou os dados de pesquisa apresentados na literatura científica publicada para obter uma compreensão mais clara dos fatos e descobertas. Dependendo das políticas de um determinado periódico, os artigos podem incluir relatórios de pesquisas originais, reanálises de pesquisas de outros, revisões da literatura em uma área específica, propostas de teorias novas, mas não testadas, ou artigos de opinião. Minha fonte: \cite{org_what_nodate}. (Tradução livre). \todo{jhcf: Faltou o ano da publicação}} }

\section{Minhas impressões iniciais sobre a ciência, por João Victor de Souza Calassio}

O ser humano sempre buscou formas de explicar os fenômenos que acontecem ao seu redor. A ciência e o método científico são formas de tentar explicar e entender esses fenômenos de forma sistemática e organizada. Graças a essas explicações e entendimentos resultantes da atividade científica, adquirimos o conhecimento necessário para moldar o mundo à nossa volta e superar desafios que nossos antepassados enfrentaram. 

A ciência é uma atividade social \citep{leal_ciencia_2016}, que gera produtos reais (\gls{PublicacaoCientifica}) que podem ser corroboradas, refutadas e/ou estendidas em trabalhos futuros pela comunidade científica.

%\addbibresource{tarefa-masathayde.bib}

\section{Minhas impressões iniciais sobre a ciência, por Marco Antônio Souza de Athayde}

A ciência não necessariamente segue um caminho contínuo, ininterrupto, previsível. Mesmo com o rigor e a eficiência da metodologia científica, ela, até o momento, é incapaz de perfeitamente modelar todos os fenômenos, naturais ou não.

Em vários momentos da história, a comunidade científica encontrou-se em crise. Isso ocorre quando o conhecimento corrente alcançou seu limite, e cientistas não conseguem criar modelos satisfatórios para solucionar os problemas mais relevantes. Nesses momentos, somente por meio de uma \gls{revolucaocientifica}, a ciência avança e supera os obstáculos intelectuais. Novos paradigmas são estabelecidos, e a ciência reorganiza-se, mais preparada e resistente.

Sem dúvida, dentre todas as áreas de atuação humana, talvez a ciência seja a que reúne os indíviduos mais preparados para uma quebra de paradigma. Realizar ciência é, pois, uma revelação de limitação e vieses pessoais. O cientista deve desapegar-se de tudo que atrapalhe a busca pelo conhecimento e estar sempre pronto para deixar de lado o conhecimento anterior, caso seja demonstrado que este não é mais adequado.

Ainda assim, a ciência está sujeita a influências deturpadoras, como interesses pessoais e financeiros, como descrito em \citep{chavalarias_whats_2017}. Por isso, a comunidade científica deve continuar a esforçar-se para manter sua integridade, buscando sempre o ideal da prática científica.

\newglossaryentry{EscritaAcademica}{
	name={Escrita Acadêmica},
    description={É um estilo específico de escrita utilizado para documentar conhecimento científico. Esse estilo de escrita utiliza padrões como os apresentados por \citep*{regina_escrita_2021}.\\
    Um guia de escrita deste padrão pode ser encontrado em \cite{noauthor_academic_2022}.\cite{jhcf: Faltou o nome do autor na referência.} }}

\newglossaryentry{BaseEmpirica}{
    name = {Base Empírica},
    description  = {\citet{ancona_lopez_consideracoes_2011} disse: ``A base empírica, assim como o conhecimento empírico, vai variar de pessoa para pessoa.  Se o mesmo método for aplicado a um mesmo objeto, por pessoas diferentes, o resultado será diferente. Este resultado é a base empírica da pesquisa. Portanto a base empírica é mutável pela mudança do pesquisador, pela mudança de métodos e, claro, pela mudança de objeto.Uma pesquisa, portanto, começa sem a base empírica em si, mas apenas com um indicativo de seus instrumentos (método e objeto).Para alguns, sobre começar a pesquisa sem base empírica, segundo primeira interpretação do assunto, entende-se que é possível você começar sem a base empírica, uma vez que você vai criar, por exemplo, questionários ou roteiros de entrevista para o levantamento dos dados que serão analisados.'' Fonte: \citep{ancona_lopez_consideracoes_2011} \todo{jhcf: Não há uma \textbf{definição} para base empírica no seu item de glossário, como seria esperado}.}}

\section{Minhas impressões iniciais sobre a ciência, por Plínio Candide Rios Mayer}

\todo{jhcf: Plínio, o seu nome e username de usuário não estão na abertura desse documento. Regularizar a situação.}

A ciência refere-se principalmente à sistematização do conhecimento. A espécie humana é composta de descendentes fracos dos primatas, pouco adaptados aos ambientes aos quais pertenciam \todo{jhcf: Fracos e pouco adaptados em que sentido? Seria bom apresentar uma referência bibliográfica para suportar esse argumento}, sobrou aos seres humanos utilizar a capacidade de acumular e produzir conhecimento para garantir a própria sobrevivência. Por muito tempo esse conhecimento foi gerenciado individualmente ou por grupos pequenos de pessoas, o advento do \gls{MetodoCientifico} permitiu um avanço exponencial do conhecimento.

Como descrito na página \citet{wikipedia_metodo_2021} o método científico permitiu a sistematização, não apenas do conhecimento, mas dos procedimentos que produzem o conhecimento permitiu a qualquer pessoa, por mais medíocre que ela seja, contribuir para o avanço da ciência.
\todo{jhcf: Rever o uso dos sinais de pontuação de frase, como ponto e vírgula.}


\section{Minhas impressões iniciais sobre a ciência, por Matheus Barbosa Santos}

A ciência tem uma importância muito grande em nossas vidas e fundamental para a sociedade moderna pois sua \gls{ProducaoCientifica} possibilitou e ainda segue a proporcionar descobertas e criações que ajudam a melhorar nossas vidas. É graças a \gls{RevolucaoCientifica} que podemos ter a medicina e seus tratamentos e prevenções de doenças que a tempos atrás eram comuns e mortais. \todo{jhcf: Não há uma única revolução científica. São várias, cada uma com suas especificidades. No caso citado, da medicina, qual teria sido a específica? Quando ocorreu? Quais foram os conhecimentos novos? Que paradigmas foram ou continuam sendo desafiados?} Também é graças a ela que temos a tecnologia e a ciência da computação a qual faço minha graduação.


\section{Minhas impressões iniciais sobre a ciência, por Isaque Augusto da Silva Santos}

A ciência tem como peça fundamental o questionamento, desde os primórdios da humanidade com o advento das primeiras correntes filosóficas, como por exemplo o racionalismo (\gls{Racionalismo}), até o momento atual. Nesse sentido, notou-se que para explicar diversos fenômenos era necessário uma metodologia auxiliadora, também chamada de método científico \citet{wikipedia_metodo_2021}, que surgiu em meados do Século 12 e ainda é utilizada hodiernamente.

Isto posto, consigo concluir que a ciência sempre esteve presente e que tem extrema importância na resolução de paradigmas (\gls{Paradigma}) e resposta a indagações. Ademais, ela é a principal responsável pelo desenvolvimento de novas ferramentas e tecnologias (\gls{tecnologia}) que podem ser utilizadas para sanar novas dúvidas.



\chapter{Análise bibliográfica sobre simulações e experimentos voltados para a influência de opniões, por Gabriel Ligoski\label{chap:bibliometria:gabrielligoski}}

\section{Planejamento do estudo}
O interesse pelo tema surgiu das recentes problemáticas envolvendo a manipulação de opnião pública, a internet como meio propagador e as consequências destes problemas.
Algo muito comum por exemplo são empresas pagarem para pessoas publicarem críticas positivas sobre seus produtos para assim influenciar pessoas a se tornarem consumidores e depois postarem positivamente sobre este produto, como demonstrado neste vídeo \cite{linus_tech_tips_its_2022}, pessoas chegam a enganarem seus sentidos para fazer parte de um grupo de opniões.
\cite{anderson_development_2014} fizeram um modelo de simulação baseado em liderança de opinião sobre enfermeiras, para agilizar o processo de aprendizagem e melhorar os resultados do hospital. A opinião de enfermeiras com autoridade, ou seja mais capacitadas, trazia segurança e influenciava nas decisões e opniões de outras enfermeiras. Neste estudo a influência foi utilizado de forma positiva.

No caso do meu trabalho, as perguntas que o nortearam foram:
\begin{itemize}
    \item É possível prever a prevalência de opniões baseado na sua popularidade?
    \item Como pode haver harmonia entre opniões divergentes?
    \item Como grupos de agentes com ideias semelhantes se formam e evoluem?
    \item Porque fragmentos de opniões tendem a desaparecer?
\end{itemize}

\section{Coleta de dados}
A coleta de dados feita usando a Scopus no dia 06 de dezembro de 2022, acessado por meio do
Portal de Periódicos da CAPES.

\subsection{Query da busca}
Foi utilizada a query: 
\begin{verbatim}
TITLE-ABS-KEY ( ( "experiment" OR "simulation" OR "multi-agent" ) AND ( "opinion seeking" OR "word of mouth" OR "opinion leader" OR "user influence" ) ) AND PUBYEAR > 2000 AND PUBYEAR < 2023 AND ( LIMIT-TO ( LANGUAGE , "English" ) )
\end{verbatim}
Fora pesquisado ao menos um entre experimento, simulação e multi-agente e ao menos um entre busca de opinião, boca a boca, lider de opinião e influência de usuário. Aplicados no título, abstract, author e keywords. Filtrando para publicações a partir do ano 2000 e em inglês.

\subsection{Registros recuperados}
Foram obtidos 1,621 documentos que encontram-se em \url{https://github.com/gabrielligoski/bibliographic-analysis-of-opinion-influence/}.
Para obter estes documentos utilizei a opção de exportar como csv todos os documentos da pesquisa e selecionei todos os campos.

\section{Análise dos dados}
\subsection{Filtros}
Foi aplicado os filtros de data e linguagem, filtrando todos os artigos dentro do período de 2000 a 2023 e em inglês.

\subsection{Análise descritiva do dataset}

As informações mais gerais sobre o dataset são as seguintes:
\begin{description}
    \item [Crescimento anual] Em média cresceu 20.21\% ao ano com ano de maior crescimento e publicações 2021.
    \item [Média de citação anual] Em média foram 14 citações, com o ano de 2003 distoante com 35 citações médias por artigo publicado.\footnote{Note que neste ano houveram apenas 4 documentos obtidos, por isso provavelmente um deles causou essa discrepancia.}.
    \item [Fonte mais relevante] A fonte com maior quantidade de artigos foi \begin{verbatim}
LECTURE NOTES IN COMPUTER SCIENCE (INCLUDING SUBSERIES LECTURE NOTES IN ARTIFICIAL INTELLIGENCE AND LECTURE NOTES IN BIOINFORMATICS)
\end{verbatim}
    \item [Campo de pesquisa mais relevante] Computadores em comportamento humano, isso é um reflexo de como as novas tecnologias principalmente redes sociais veem impactando o comportamento humano.
    \item [Autores mais relevantes] Liu Y e Wang X lideram o quadro de documentos publicados com 22 cada.
    \item [Países lideres em citação] USA lidera com 18311 citações, logo em seguida vem China com 3442, Koreia com 1994, Alemanha com 1871 e Reino Unido com 1498. \footnote{O número de citações dos USA provavelmente se deve ao filtro de linguagem imposto.}.
\end{description}

\subsection{Evolução da Produção Científica}

\begin{figure}
    \centering
    \includegraphics[width=1\textwidth]{exploratory-data-analysis/gabrielligoski/PesqBibliogr/ColorPatches/AnnualScientificProduction-2022-12-06.png}
    \caption{Evolução da produção científica no dataset}
    \label{fig:evol:anual:gabrielligoski}
\end{figure}

A produção de Conteúdo científico deste tema vem crescido a uma taxa acelerada indicando uma possível tendência de popularidade para os próximos anos e que é um tema pertinente principalmente se tratando de temas muito discutidos como o impacto/influência de redes socias para disciminação de opiniões, com uso de ferramentas como twitter.

\subsection{Evolução da Produção Científica}

\begin{figure}
    \centering
    \includegraphics[width=1\textwidth]{exploratory-data-analysis/gabrielligoski/PesqBibliogr/ColorPatches/AverageArticleCitationPerYear-2022-12-06.png}
    \caption{Média de citações anuais no dataset}
    \label{fig:cit:anual:gabrielligoski}
\end{figure}

O gráfico tem um pico em 2003 por se tratar de um baixo número de artigos que foram muito citados, porém é interessante ver que mesmo com um crescente número de documentos cientificos o número se manteve constante após 2003 indicando um crescimento orgânico do interesse pelo tema.


\subsection{Relação palavra-chave a país}

\begin{figure}
    \centering
    \includegraphics[width=1\textwidth]{exploratory-data-analysis/gabrielligoski/PesqBibliogr/ColorPatches/threeFieldPlot.png}
    \caption{Análise do interesse por país e palavra-chave}
    \label{fig:cit:anual:gabrielligoski}
\end{figure}

Analisando o gráfico podemos ver uma tendência do interesse de cada país por algum campo específico, os USA e Reino Unido parecem buscar mais o interesse econômico destes estudos com foco em opinião, usuários, resultados e consumidores. Enquanto a China, Australia e Alemanha tem um interesse maior em efeitos sociais com foco em social, influência e informação.


\subsection{Relevância de temas}

\begin{figure}
    \centering
    \includegraphics[width=1\textwidth]{exploratory-data-analysis/gabrielligoski/PesqBibliogr/ColorPatches/MostRelevantSources-2022-12-06.png}
    \caption{Relevância de temas pelo abstract}
    \label{fig:cit:anual:gabrielligoski}
\end{figure}

Analisando o gráfico podemos ver que boa parte se deve ao quesito de simulação e modelo, pois a maioria vem de ciência da computação.

\subsection{Autores}

\begin{center}
\begin{tabular}{||c c ||} 
\hline
 Author & Artigos\\ [0.5ex] 
 \hline\hline
LIU Y	& 22\\ \hline
WANG X	& 22\\ \hline
ZHANG Y	& 19\\ \hline
LI Y	& 18\\ \hline
WANG Y	& 15\\ \hline
WANG H	& 14\\ \hline
LI H	& 13\\ \hline
WANG J	& 13\\ \hline
ZHANG J	& 13\\ \hline
CHEN Y	& 12\\ \hline
\end{tabular}
\end{center}

Vendo a tabela com os 10 autores com maior número de artigos podemos ver um grande interesse e investimento  por parte dos chineses.

\subsection{Disputa demográfica}

\begin{center}
\begin{tabular}{||c c c ||} 
\hline
 Country & Year & Articles\\ [0.5ex] 
 \hline\hline
CHINA	& 2022	& 1590\\ \hline
CHINA	& 2021	& 1391\\ \hline
USA	& 2022	& 1205\\ \hline
CHINA	& 2020	& 1181\\ \hline
USA	& 2021	& 1054\\ \hline
CHINA	& 2019	& 1022\\ \hline
USA	& 2020	& 949\\ \hline
CHINA	& 2018	& 891\\ \hline
USA	& 2019	& 846\\ \hline
USA	& 2018	& 734\\ \hline
\end{tabular}
\end{center}

Ordenando pelo número de artigos podemos ver que a China e os USA vem liderando os investimentos no campo, com o interesse de políticos no aspecto da opinião pública é compreensível que haja um bom investimento por países com grande interesse no controle mundial das opiniões e cultura.


\begin{center}
\begin{tabular}{||c c c ||} 
\hline
From &	To &	Frequency\\ [0.5ex] 
 \hline\hline
CHINA	& USA	& 74\\ \hline
CHINA	HONG & KONG	& 23\\ \hline
CHINA	& AUSTRALIA	& 20\\ \hline
USA	& KOREA	& 19\\ \hline
USA	& CANADA	& 17\\ \hline
USA	UNITED & KINGDOM	& 13\\ \hline
CHINA	UNITED & KINGDOM	& 8\\ \hline
USA	HONG & KONG	& 8\\ \hline
USA	& NETHERLANDS	& 8\\ \hline
GERMANY	& DENMARK	& 7\\ \hline\hline
 Country & Year & Articles\\ \hline
\end{tabular}
\end{center}

Ordenando pela frequência vemos que China colabora bastante com os USA, porém isso não é reciproco, possivelmente por conta da linguagem dos documentos que não obteve documentos chineses.

\section{Análise dos documentos}
\subsection{Previsibilidade}
Em \cite{aral_creating_2011} é estudado a influência de empresas em contágio social, buscando a criação de um "boca a boca" eficiente, que efetivamente é criar uma opinião sobre uma marca ou produto. Neste estudo é possível prever até certo ponto o sucesso de uma "opinião" criada por uma marca com base no recurso utilizado para viraliza-la.

\subsection{Ideias contrárias}
Em \cite{bail_exposure_2018} criam-se robôs para postar no twitter opiniões divergentes nos USA, um para democratas e outro para republicanos, o comportamento dos robôs era responder automaticamente posts de seus pares com ideias  contrárias, ao final do estudo pode-se notar que o robô conseguiu mudar um pouco da opinião de seus seguidores acumulados. Criando de certa forma uma pequena tendencia de harmonia entre ideias opostas.

\subsection{Fragmentos e minorias}
Em \cite{chen_online_2011} foi estudado a influência da opinião de outros consumidores pelo boca a boca e avaliações do produto, buscando o impacto na opinião de outros consumidores, notaram que as avaliações positivas impactaram o número de vendas porém as avaliações negativas tiveram um impacto negligente. Possivelmente a vontade das pessoas de fazer parte da opinião de um grupo grande impacta nestas decisões, assim as pessoas avaliam positivamente os produtos pelo que lhes foi dito. Assim como acontece no vídeo \cite{linus_tech_tips_its_2022}, dessa forma fragmentos tendem a desaparecer.

\newglossaryentry{MetodoCientifico}{
        name = {Método Científico}, 
        description  = {O método científico é uma estratégia de prova de um estudo para validação  como um conhecimento verdadeiro, ou seja, é um conjunto de etapas a serem seguidas para que um estudo seja considerado científico. Tais etapas estão separadas em 6 categorias, que vão de questões teóricas à analise práticas, o que leva dessa forma a uma conhecimento concreto por qual  se formula leis e teorias científicas. Fonte: \cite{wikipedia_metodo_2021}. \todo{jhcf: quais categorias seriam essas? Essa definição não foi encontrada por mim na referência citada.}}}

\newglossaryentry{Paradigma}{
name = {Paradigma},
description = {Paradigma é um conceito das ciências e da epistemologia que define um exemplo típico ou modelo de algo. É a representação de um padrão a ser seguido. É um pressuposto filosófico, matriz, ou seja, uma teoria, um conhecimento que origina o estudo de um campo científico; uma realização científica com métodos e valores que são concebidos como modelo; uma referência inicial como base de modelo para estudos e pesquisas. Fonte: \citep{noauthor_paradigma_nodate}. \todo{jhcf: Referência sem autor nem data.}
}}

\section{Minhas impressões iniciais sobre a ciência, por André Cássio Barros de Souza}

Vejo a ciência como a forma que os humanos possuem para tentar explicar os fenômenos que ocorrem no universo, buscando o conhecimento para entender o ambiente que o cerca. Uma atividade de extrema importância para o desenvolvimento da sociedade, visto que a geração de \gls{tecnologia} tem como base o uso do ciência.

A humanidade busca aprimorar como essa atividade é realizada, apresentada e registrada, a \gls{ComunidadeCientifica} mostra como a ciência agrupa pessoas de interesse em comum para a realização da \gls{ProducaoCientifica}. Esses interesses e assuntos podem ser classificados na forma de uma \gls{Hierarquia}, a fim de tentar organizar as áreas do conhecimento.

\todo{jhcf: A definição de hierarquia apresentada no glossário não está coerente com o seu uso aqui. Sugiro Melhorar a definição.} 

\section{Minhas impressões iniciais sobre a ciência, por Regina Emy Da Nóbrega Kamada}

\todo{jhcf: Regina, Seu nome completo e username não se encontram no início deste documento. Corrigir essa situação.}

A ciência é o desenvolvimento cético e empírico da compreensão sobre o universo. Isto é, o entendimento sobre fenômenos naturais em todas as escalas, como o nascimento de estrelas na nebula carina (em maior escala espacial), os impulsos eletromagnéticos em neurônios (em menor escala espacial) e as leis da física (em todas escalas, por definição).  

Qualquer conhecimento produzido pelo \gls{MetodoCientifico}, fundamentado na \gls{BaseEmpirica} e discutido no \gls{Paradigma} do  \gls{Racionalismo} é uma ciência— por mais que hajam ciências fora de tal regra, dado que muitos conhecimentos possuem dificuldades de execução experimental, seja pela ética, pela incapacidade de fisicamente executá-los ou conceber testes precisos e claros com valores argumentativos. Por exemplo, \citet{mccown_schellings_nodate} é capaz de virtualmente simular um modelo de segregação racial, porém, ao mesmo tempo que é questionável se os mesmos resultados seriam reproduzidos em simulação com pessoas, o campo de engenharia social possui notórios empecilhos para confecção de experimentos éticos com valores precisos para a argumentação científica, mas permanece uma engenharia com produção científica de grande relevância no mercado corporativo.

\section{Minhas impressões iniciais sobre a ciência, por Lucas de Oliveira Silva}

Após alguns anos convivendo com vários estudos sobre os mais variados tipos durante o ensino médio em uma universidade federal, em diversas matérias, e agora na UnB tive contato com esse assunto poucas vezes e acredito que apenas agora posso começar a ter minha concepção sobre o que é a ciência e meus primeiros pensamentos em relação a ela. A ciência se baseia e consiste em estudo e a forma como estudamos qualquer tipo ou área do conhecimento, afim de expandir ou aprimorar o conhecimento humano, como já vi no ensino médio, em filosofia, tudo se origina da própria filosofia, desde a matemática até sociologia, tudo se tem origem na antiga e única filosofia, que era a unificação de todo o conhecimento, que devido ao grande conhecimento, foi ramificada e denominada em diferentes áreas, é interessante notar que a ciência já tem sua complexidade desde o seu próprio conceito e aplicação, com métodos que precisam ser bem definidos e dependendo também da \gls{Universalidade}, reproducibilidade e projetização de sua aplicação, dentre outros conceitos que definem a ciência e o método cientifico.

Devido a complexidade de seu conceito e aplicação, devemos tomar muito cuidado com a maneira que tudo é feito, para acharmos a melhor teoria possível, com provas validas, e mesmo sendo assim durante toda a história tivemos varias repaginações de teorias que foram invalidadas por outras descobertas, um conceito que ajudada na construção de tudo é a cientometria \citep{vinkler_evaluation_2010}.
\todo{jhcf: Cientometria é um conceito? Ajuda de que forma?} 

\newglossaryentry{CienciaAplicada}{
	name={Ciência Aplicada},
	description={A Ciência Aplicada é um ramo da ciência que busca a aplicação prática das teorias e conhecimentos científicos. As ciências que fazem participam\todo{jhcf: parte...} dessa modalidade procuram a resolução de problemas reais a partir de métodos e conhecimentos científicos, que podem ter sidos produzidos por outras áreas. Dentro das ciências aplicadas, a computação é uma das que o interesse mais cresce atualmente. Muitas empresas se dedicam muito em pesquisas na área da ciência da computação voltadas para o desenvolvimento (P\&D) a fim de desenvolver tenologias inovadoras ou novas soluções a partir do conhecimento científico adquirido. Fonte: \citep{international_student_applied_2022}  }}


\newglossaryentry{bibliometria}{
    name={Bibliometria},
    description={A bibliometria é o estudo dos aspectos quantitativos da produção, disseminação, socialização e evidenciação da informação registrada. Fonte: \citep[p. 3]{ribeiro_bibliometria_2018} }}

\newglossaryentry{RevolucaoCientifica}{
    name={Revolução Científica},
    description={Questionar dogmas consagrados, a ver o progresso da ciência não é dado como o acúmulo gradativo de novos dados gnosiológicos, e sim como um processo contraditório marcado pelas revoluções do pensamento científico. Tais revoluções são definidas como o momento de desintegração do tradicional numa disciplina, forçando a comunidade de profissionais a ela ligados a reformular o conjunto de compromissos em que se baseia a prática dessa ciência \cite{kuhn_estrutura_2001} }
}

% Com base no exemplo acima, cada estudante deve usar, ao final do arquivo, uma tag input para as suas impressões pessoais

