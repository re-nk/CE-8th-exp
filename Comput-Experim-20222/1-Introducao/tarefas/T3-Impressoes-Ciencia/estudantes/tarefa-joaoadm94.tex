\section{Minhas impressões iniciais sobre a ciência, por João Antônio Desidério de Moraes}

O texto motivador enseja algumas percepções sobre a ciência. Em primeiro lugar, a compreensão de ciência como aliança entre propósito e método científico (\gls{MetodoCientifico})     . O esforço científico tem finalidade explicativa, descritiva e preditiva dos fenômenos observados no mundo. Esse ânimo, por sua vez, é compartilhado por diversas outras disciplinas humanas. Assim sendo, o que separa ciência das explicações de fenômenos oferecidas pela fé ou por mitologias é seu caráter metódico. Não basta sugerir uma teoria; é necessário seguir regras e princípios já estabelecidos para caracterizar o conhecimento científico. Isso permite fazer ciência até mesmo sobre a própria ciência, visto podermos formular hipóteses sobre a produção científica utilizando seus métodos. Podemos citar a bibliometria (\gls{bibliometria}), uma ciência sobre o fenômeno da publicação científica(\gls{PublicacaoCientifica}), como exemplo deste tipo de ciência.

A segunda impressão relevante sobre a ciência tem a ver com seu caráter social. Nesse contexto, a ciência é feita por humanos e se aproveita do caráter dialógico da cultura humana. São formadas comunidades científicas (\gls{ComunidadeCientifica}) baseadas no comum interesse por um assunto para organizar a produção científica, onde esta se fortalece por meio da participação crítica dos envolvidos. Justamente por seu caráter humano e social, o conhecimento científico não se pretende absoluto. São reconhecidas as limitações das teorias e ferramentas humanas, fator responsável por crises e revisões regulares do corpo científico existente.
