\section{Minhas impressões iniciais sobre a ciência, por Isaque Augusto da Silva Santos}

A ciência tem como peça fundamental o questionamento, desde os primórdios da humanidade com o advento das primeiras correntes filosóficas, como por exemplo o racionalismo (\gls{Racionalismo}), até o momento atual. Nesse sentido, notou-se que para explicar diversos fenômenos era necessário uma metodologia auxiliadora, também chamada de método científico \citet{wikipedia_metodo_2021}, que surgiu em meados do Século 12 e ainda é utilizada hodiernamente.

Isto posto, consigo concluir que a ciência sempre esteve presente e que tem extrema importância na resolução de paradigmas (\gls{Paradigma}) e resposta a indagações. Ademais, ela é a principal responsável pelo desenvolvimento de novas ferramentas e tecnologias (\gls{tecnologia}) que podem ser utilizadas para sanar novas dúvidas.
