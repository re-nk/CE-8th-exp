\section{Minhas impressões iniciais sobre a ciência, por Fernando Ferreira Cordeiro}

A ciência nasceu junto ao desenvolvimento dos seres humanos. Está ligada essencialmente a sua evolução, é ela que satisfaz a nossa curiosidade e necessidade de conhecimento. Não é erro dizer que a ciência fez e faz a espécie humana evoluir. Por meio dos seus métodos (\gls{MetodoCientifico}) e instrumentos (\gls{tecnologia}) para construir um entendimento sobre a realidade através do pensamento racional e da observação sistemática dos fatos, a ciência nos permitir \todo{permite...} ver além do que os olhos são capazes de ver, trazendo a imaginação para o mundo físico. Os frutos de suas pesquisas presente está \todo{estão presentes...} desde quando acordamos até quando vamos dormir, se tornou está \todo{tornaram...} algo tão essencial nos dias de hoje que é impossível pensar o mundo sem ela. 

0 desenvolvimento de um país está diretamente relacionado ao investimento na ciência. Instituições esportivas, automobilismo, instrumentos hospitalares, engenharia aeroespacial, tudo isso se explica com a quase hegemonia de países desenvolvidos nesse setores, eles não atingiram esse nível por acaso. Foram adotadas politicas e investimentos em desenvolvimento científico e tecnológico em diversas áreas, sempre mitigando superar uma necessidade.       

está \todo{estão presentes...} 