\section{Minhas impressões iniciais sobre a ciência, por Marcelo Aiache Postiglione}

A ciência pode representar tanto uma área de conhecimento quanto um processo. Em meu entendimento, a ciência como área do conhecimento é o conjunto de toda a sabedoria adquirida sobre aquele escopo específico (Ciência da computação, Ciências biológicas, Ciências Sociais, etc). Como processo, a ciência é uma atividade humana que segue passos sistemáticos com o objetivo de adquirir ou aprofundar o conhecimento sobre algo. Esses passos são usados, por exemplo, para levantar dados, criar uma \gls{Hipotese} ou formular uma nova teoria. É uma atividade guiada pelo método científico e outros princípios, que visa estudar, entender e registrar os mais diversos fenômenos do universo e assim, tentar  interligar fatos isolados com o objetivo de promover uma melhor compreensão sobre o mundo e a natureza. 

Essa busca sistemática pelo conhecimento, nos permitiu grandes avanços e sua contribuição para a sociedade é inegável. Por exemplo, durante a pandemia da COVID-19 tivemos vários estudos importantes para entendermos melhor como a doença se manifestava e até mesmo suas implicações no coração \cite{strabelli_covid-19_2020}, além de todos os trabalhos que culminaram na criação das vacinas.