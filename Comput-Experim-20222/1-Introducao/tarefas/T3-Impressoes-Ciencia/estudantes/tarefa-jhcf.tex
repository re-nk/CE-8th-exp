\section{Minhas impressões iniciais sobre a ciência, por Jorge Henrique Cabral Fernandes}


Tendo já uma vivência de algumas décadas sobre o que é a ciência, abordarei de modo mais específico minhas impressões sobre a Ciência da Computação \citep{baldwin_three-fold_1994}, que é uma ciência específica dentre as várias existentes. A Ciência da Computação apresenta-se muito relacionada com a filosofia da ciência \citep{floridi_blackwell_2004}, ou como uma meta-ciência (A filosofia é um campo do pensamento humano que é base para a ciência). Um exemplo que comprova essa versatilidade - ou até mesmo universalidade - da Ciência da Computação é o emprego da computação para a criação de simulações como método de pesquisa empírica, isso é o uso de simulações computacionais como base para a pesquisa fundamentada em coleta e análise de dados \citep{tedre_experiments_2014}. A versatilidade do computador, sendo uma Máquina de Turing, possibilita o seu uso para gerar infinitas plataformas empíricas para coleta de dados, pois os computadores são prodigiosos na coleta, cálculo e geração de dados. Assim sendo, pode-se usar o computador para o desenho e execução de quase todo tipo de experimento científico, em quaisquer dos outros campos do conhecimento. Situações mais específicas podem ser vistas no campo da \gls{EA}.

Posto isso, refletir sobre o uso de simulações computacionais possibilita também refletir muito sobre o que é a ciência, e portanto, possibilita uma abordagem filosófica.
