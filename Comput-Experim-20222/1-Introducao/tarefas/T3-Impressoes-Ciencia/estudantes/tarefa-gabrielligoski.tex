\section{Minhas impressões iniciais sobre a ciência, por Gabriel Ligoski}

Ciência para mim é principalmente relacionado a criação, avanço e reproducibilidade de experimentos, tendo como base experimentos empiricos.
\begin{itemize}
  \item Criação: É o primeiro passo para avanços cientificos, pois eles precisam ser formulados e pensados antes de existirem e serem trabalhados, então esta é uma etapa de planejamento, cria-se expectativas sobre o experimento, limites e como será conduzido.
  \item Avanço: Esta etapa faz parte de um acumulo de conhecimento sobre o tema, utilizando o conhecimento já documentado é possível ter novas perspectivas e novos resultados para temas já explorados, assim concretizando cada vez mais o conhecimento sobre o tema.
  \item \gls{Reproducibilidade}: para que haja avanços os experimentos devem ser reproduziveis, caso contrário é impossível concretizar algum conhecimento, afinal se não é possível reproduzir um erro ele realmente existe? Por isso é importante que os passos para criação de conhecimento sejam seguidos rigorosamente.
\end{itemize}

Dito isto é interessante pensar que futuramente haja possibilidade de criar novos universos inteiramente para reproduzir um experimento fielmente que tenham consistência.
\todo{jhcf: nome e username não estão apresentados no início deste documento. Solucionar o problema.}