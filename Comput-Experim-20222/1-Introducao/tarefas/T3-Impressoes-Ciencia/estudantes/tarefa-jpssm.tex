\section{Minhas impressões iniciais sobre a ciência, por João Pedro de Sousa Soares Martins}



A Ciência é basicamente baseada na busca dos seres humanos por explicações sobre o universo. Em universo pode-se incluir tanto o universo físico, que inclui os corpos celestes e seus seres, como nós humanos, quanto os fenômenos que nele ocorrem, que pode ser simples e objetivos como uma reação química entre duas moléculas ou complexos e subjetivos como o desenvolvimento das várias civilizações humanas. Toda ciência procura descrever os objetos de estudos de sua área e explicar os fenômenos em que esses participam, apontando as causas, meios e consequências. Nesse sentido a ciência procura a "verdade" (\citep{decourt_atividade_2022}), ou pelo menos que mais se aproxima disso.

Apesar do objetivo básico da ciência em geral, que é a busca pela verdade, as diferentes áreas da Ciência surgem e se desenvolvem com diferentes finalidades. Algumas áreas, como a Astronomia são motivadas por puramente por questões científicas. Seus estudiosos fazem pesquisas e produzem conhecimento somente para que se entenda mais o universo observável, pode-se dizer que é uma motivação puramente científica. Por outro lado muitas áreas, como as Ciências Aplicadas (\gls{CienciaAplicada}), possuem interesses econômicos que motivam o seu desenvolvimento. A computação é um forte exemplo. Muito do que se produz atualmente provem de uma grande demanda existente no mercado para o desenvolvimento de tecnologias e o aperfeiçoamento de produtos. Nesse quesito embora essas produções cientificas sejam válidas, e não deixem de ser ciência, não se pode dizer mais que o motor principal é a busca pela verdade. 
