\section{Minhas impressões iniciais sobre a ciência, por Thiago Masera Tokarski}

Encaro a ciência como uma forma sistemática de adquirir e registrar conhecimento. Conhecimento este que serve de base para a resolução de problemas, até mesmo os problemas ainda desconhecidos. Assim sendo, a ciência se apresenta como um investimento a longo prazo para seus interessados. Realizando análise sobre os registros científicos observa-se que as potenciais globais são as maiores produtoras de publicações científicas, pois, apresentam os maiores investimentos no âmbito acadêmico. Pesquisas científicas geram novos conhecimentos que por sua vez proporcionam bases para inovações e estimulam atividades econômicas inéditas \cite{ost_dynamics_2019}. 

Como resultado dessas diversas pesquisas temos a compilação dos registros científicos nos \gls{scientific_journals} que representam a forma mais vital de compartilhamento do conteúdo científico. O contexto da ciência no pós pandemia deixou claro a importância de uma comunicação mais aberta e acessível do conteúdo científico, oque pode ser feito talvez através dos \gls{scientific_journals}.

