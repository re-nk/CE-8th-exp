\section{Minhas impressões iniciais sobre a ciência, por Plínio Candide Rios Mayer}

\todo{jhcf: Plínio, o seu nome e username de usuário não estão na abertura desse documento. Regularizar a situação.}

A ciência refere-se principalmente à sistematização do conhecimento. A espécie humana é composta de descendentes fracos dos primatas, pouco adaptados aos ambientes aos quais pertenciam \todo{jhcf: Fracos e pouco adaptados em que sentido? Seria bom apresentar uma referência bibliográfica para suportar esse argumento}, sobrou aos seres humanos utilizar a capacidade de acumular e produzir conhecimento para garantir a própria sobrevivência. Por muito tempo esse conhecimento foi gerenciado individualmente ou por grupos pequenos de pessoas, o advento do \gls{MetodoCientifico} permitiu um avanço exponencial do conhecimento.

Como descrito na página \citet{wikipedia_metodo_2021} o método científico permitiu a sistematização, não apenas do conhecimento, mas dos procedimentos que produzem o conhecimento permitiu a qualquer pessoa, por mais medíocre que ela seja, contribuir para o avanço da ciência.
\todo{jhcf: Rever o uso dos sinais de pontuação de frase, como ponto e vírgula.}
