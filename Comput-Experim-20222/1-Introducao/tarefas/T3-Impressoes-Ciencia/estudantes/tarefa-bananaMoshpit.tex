\section{Minhas impressões iniciais sobre a ciência, por Regina Emy Da Nóbrega Kamada}

\todo{jhcf: Regina, Seu nome completo e username não se encontram no início deste documento. Corrigir essa situação.}

A ciência é o desenvolvimento cético e empírico da compreensão sobre o universo. Isto é, o entendimento sobre fenômenos naturais em todas as escalas, como o nascimento de estrelas na nebula carina (em maior escala espacial), os impulsos eletromagnéticos em neurônios (em menor escala espacial) e as leis da física (em todas escalas, por definição).  

Qualquer conhecimento produzido pelo \gls{MetodoCientifico}, fundamentado na \gls{BaseEmpirica} e discutido no \gls{Paradigma} do  \gls{Racionalismo} é uma ciência— por mais que hajam ciências fora de tal regra, dado que muitos conhecimentos possuem dificuldades de execução experimental, seja pela ética, pela incapacidade de fisicamente executá-los ou conceber testes precisos e claros com valores argumentativos. Por exemplo, \citet{mccown_schellings_nodate} é capaz de virtualmente simular um modelo de segregação racial, porém, ao mesmo tempo que é questionável se os mesmos resultados seriam reproduzidos em simulação com pessoas, o campo de engenharia social possui notórios empecilhos para confecção de experimentos éticos com valores precisos para a argumentação científica, mas permanece uma engenharia com produção científica de grande relevância no mercado corporativo.