\section{Minhas impressões iniciais sobre a ciência, por Gustavo Rodrigues Gualberto}

Durante toda a minha vida e minha experiência obtida através dos âmbitos acadêmicos, a minha percepção acerca da ciência parte do princípio que baseia essa área - o \gls{MetodoCientifico}. Então, basicamente o \gls{Paradigma} desta provêm do estudo baseado em provas e refutações empíricas, ou seja, para que algo seja admitido como verdade, faz-se mister que esta possa ser realizado no plano terreno, por isso, é impossível provar questões do campo espiritual, sobrenatural e religioso usufruindo de tal método, pois tais só se manifestam no plano metafísico. Ademais, constitui-se imprescindível evitar acreditar indubitavelmente no que está comprovado atualmente pela ciência, porque a verdade que acreditamos no momento pode ser refutada no futuro e está limitada às nossas limitações físicas e mentais do corpo humano, embora eu creia que o \gls{MetodoCientifico} seja a melhor maneira de provar tudo que está presente no campo físico de estudo. Por fim, espero que a ciência continue ajudando a raça humana em sua longa jornada pela eternidade e se porventura for possível e conveniente, que eu faça parte dessa comunidade e possa contribuir para a sua excelência.

\citep{noauthor_paradigma_nodate}
\todo{jhcf: melhorar a qualidade dessa referência bibliográfica.}