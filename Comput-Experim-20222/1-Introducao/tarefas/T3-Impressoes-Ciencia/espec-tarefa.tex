\chapter{Tarefa 3: Impressões Iniciais Sobre a Ciência}

\section{Instruções}

\subsection{Fundamentos}

O que é a Ciência? Vamos refletir sobre o que você aprendeu com as aulas sobre ciência?

\section{A Tarefa}

Cada estudante, a partir do que já sabe e leu, deve criar no diretório a seguir, um arquivo contendo uma seção de capítulo com o seu nome, a fim de apresentar até dois parágrafos de texto de sua própria autoria, apresentando as suas \textbf{Impressões iniciais sobre o que é a Ciência}:
\begin{verbatim}
1-Introducao /tarefas /T3-Impressoes-Ciencia /estudantes    
\end{verbatim}

O nome do arquivo á ser criado deve ter a forma:

\begin{verbatim}
tarefa-<githubusername>.tex
\end{verbatim}

O texto do arquivo deve seguir o modelo da tarefa exemplo do professor, em:
\begin{verbatim}
1-Introducao /tarefas / T3-Impressoes-Ciencia /estudantes /tarefa-jhcf.tex
\end{verbatim}

No registro de suas impressões iniciais no texto da tarefa, você precisa:
\begin{itemize}
    \item Usar um ou mais dos itens do glossário, incluído o que você criou na tarefa descrita em \ref{tarefa:glossario};
    \item Usar uma ou mais citações a referências bibliográficas usando a \textit{tag} \verb|\citet{}| ou \verb|\citep{}|. Não use urls com a \textit{tag} \verb|\url{}|. Note que todas as referências citadas devem estar registradas no arquivo \verb|RESIC.bib|, gerado a partir da bibliografia no grupo RESIC em \url{https://www.zotero.org/groups/2465026/resic}, ao qual você deve ter acesso, como feito em tarefa anterior.
\end{itemize}

Depois de criar o texto do arquivo com o registro de suas impressões sobre a ciência, faça o \textit{input} do mesmo no arquivo:
\begin{verbatim}
1-Introducao /tarefas /T3-Impressoes-Ciencia /estudantes /main.tex
\end{verbatim}

\subsection{Compilando a Tarefa}

Após fazer o \textit{input} do conteúdo produzido por você, faça os ajustes necessários para garantir que a compilação do \LaTeX~ ocorre sem introdução de erros ou novos \textit{warnings}.

\subsection{Entregando a Tarefa no Github}

Quando a compilação da sua entrega estiver satisfatória, use a interface do Overleaf para enviar suas contribuições para o repositório \textbf{origin}, com a opção:

\begin{verbatim}
Menu -> Sync -> github -> Push Overleaf Changes do Github
\end{verbatim}

Informe no \textit{commit} a seguinte mensagem: 
\begin{verbatim}
Tarefa 3 - <Seu Nome Completo> - <githubusername>
\end{verbatim}

Após o \textit{commit} bem sucedido, use os comandos \verb|git pull; git log| ou equivalente, para ver o \textit{tag} de seu \textit{commit} na lista de \textit{commits} no repositório \textbf{origin}. Informe esse \textit{tag} no texto online desta tarefa. 
Veja um exemplo a seguir:

\begin{verbatim}
commit 3af590b7fc3b06ec4375fb7eaba3b07abb21cda1 
     (HEAD -> main, origin/main, origin/HEAD)

Author: JORGE H C FERNANDES <jhcf@unb.br>

Date:   Tue Jan 25 16:18:27 2022 -0300

    Tarefa 3 - Jorge Henrique Cabral Fernandes - jhcf
\end{verbatim}

\subsection{Registrando a entrega da Tarefa no Aprender}

O \textit{tag} do \textit{commit} deve ser informado no texto online da tarefa no Moodle, como exemplo:
\begin{verbatim}
3af590b7fc3b06ec4375fb7eaba3b07abb21cda1
\end{verbatim}

\subsection{Pontuação pela execução da  Tarefa}

A sua resposta a essa atividade vale até 3\% da pontuação total da disciplina.
